\documentclass[11pt,letterpaper,reqno]{amsart}
\usepackage{tikz}
\usetikzlibrary{positioning, shapes.geometric, arrows.meta, calc, positioning}
\usepackage{amssymb}
\usepackage{amsmath}
\usepackage{amsthm}
\usepackage{amsfonts}
\usepackage{mathtools}
\usepackage{bbm}
\usepackage{enumitem} 
\usepackage{pgfplots}
\pgfplotsset{compat=1.18} 
\usepackage{booktabs}
\usepackage{graphicx}
\usepackage[T1]{fontenc}
\usepackage{doi}
\usepackage{float} 
\addtolength{\hoffset}{-1.5cm}\addtolength{\textwidth}{3cm}
\addtolength{\voffset}{-1cm}\addtolength{\textheight}{2cm}

\usepackage{bookmark}
\usepackage{hyperref}
\hypersetup{pdfstartview={FitH}}
\newcommand{\C}{\mathbb{C}}
\newcommand{\cE}{\mathcal{E}}
\newcommand{\norm}[1]{\lVert #1 \rVert}
\newcommand{\abs}[1]{| #1 |}
\newcommand{\bv}{\mathbf{v}}
\newcommand{\bw}{\mathbf{w}}
\newcommand{\tr}{\operatorname{Tr}}
\DeclareMathOperator{\rank}{rank}

\newtheorem{thm}{Theorem}[section]
\newtheorem{lem}[thm]{Lemma}
\newtheorem{prop}[thm]{Proposition}
\newtheorem{cor}[thm]{Corollary}
\newtheorem{claim}{Claim}
\newtheorem{ques}[thm]{Question}
\newtheorem{prob}[thm]{Problem}
\newtheorem{conj}[thm]{Conjecture}
\theoremstyle{definition}
\newtheorem{exm}[thm]{Example}
\newtheorem{remark}[thm]{Remark}
\newtheorem{defn}[thm]{Definition}
\numberwithin{equation}{section}
\newcommand{\N}{\mathbb{N}}
\newcommand{\taufunc}{\tau}
\newcommand{\omegap}{\omega}
\newcommand{\ord}{\operatorname{ord}}
\newcommand{\R}{\mathbb{R}}        % real numbers
\newcommand{\E}{\mathbb{E}}        % expectation
\newcommand{\Var}{\mathrm{Var}}    % variance
\newcommand{\Cov}{\operatorname{Cov}}
\newcommand{\PP}{\mathbb{P}}       % probability
\newcommand{\eps}{\varepsilon}     % epsilon
\newcommand{\ind}{\mathbf{1}}      % indicator function
\newcommand{\seq}[1]{\left(#1\right)} % sequence
\makeatother

\begin{document}

\title{A Note on Erd\H{o}s Problem~\#856}


\author[Q.~Tang]{Quanyu Tang}
\date{\today}

\address{School of Mathematics and Statistics, Xi'an Jiaotong University, Xi'an 710049, P. R. China}
\email{tang\_quanyu@163.com}

% \subjclass[2020]{}

% \keywords{}

% \begin{abstract}

% \end{abstract}

\maketitle


\section{Introduction}

Let $k\ge 2$ and $N\ge 1$ be integers. Following Erd\H{o}s~\cite{Er70}, we are interested in
the structure of subsets $A\subset\{1,\dots,N\}$ which avoid ``LCM patterns'' of
size $k$. More precisely, we say that $A$ \emph{contains an LCM-$k$-tuple} if
there exist distinct $a_1,\dots,a_k\in A$ such that all pairwise least common
multiples coincide:
\[
\operatorname{lcm}(a_i,a_j) = \operatorname{lcm}(a_1,a_2)
\qquad\text{for all }1\le i<j\le k.
\]

In~\cite[p.~124]{Er70}, Erd\H{o}s introduced the counting function
$F(k,x)$, defined as the largest integer $s$ for which one can find
\[
1 \le a_1 < \cdots < a_s \le x
\]
with the property that no $k$ of the $a_i$ have pairwise equal least common
multiple. He originally conjectured that $F(k,x)=o(x)$ for every $k\ge 3$, but
later showed that this is false for $k\ge 4$, see~\cite[Theorem~1]{Er70}.
In particular, for every $k\ge 4$ there exists $\delta_k>0$ such that
\[
F(k,x) \;\ge\; \delta_k x
\]
for all sufficiently large $x$.

In the same paper, Erd\H{o}s also considered the \emph{harmonic} analogue of
this problem. He showed in~\cite[Eq.~(19)]{Er70} that there is a constant $c_k>0$ such that, whenever
$x$ is sufficiently large and $a_1<\dots<a_s\le x$ satisfy
\begin{equation}\label{eq:Erdos-threshold}
\sum_{i=1}^s \frac{1}{a_i} \;>\; c_k \log x,
\end{equation}
then there must exist $k$ of the $a_i$ which have pairwise equal least common
multiple. 

Motivated by this, it is natural to isolate the harmonic problem
\[
f_k(N)
:= \max\Biggl\{\,\sum_{n\in A}\frac{1}{n} :
A\subset\{1,\dots,N\}\ \text{contains no LCM--$k$-tuple}\,\Biggr\}.
\]Erd\H{o}s' argument (see \cite[p.~127]{Er70}) shows that
\[
f_k(N) \;\ll_k\; \frac{\log N}{\log\log N}.
\]
However, very little is known about lower bounds for $f_k(N)$.


The following problem appears as Problem~\#856 on Bloom’s Erd\H{o}s Problems website~\cite{EP856}.
\begin{prob}
Let $k\geq 3$ and $f_k(N)$ be the maximum value of $\sum_{n\in A}\frac{1}{n}$, where $A$ ranges over all subsets of $\{1,\ldots,N\}$ which contain no subset of size $k$ with the same pairwise least common multiple.

Estimate $f_k(N)$.
\end{prob}

It is immediate from the definition that we have the monotonicity
\[
k\ge 3 \quad\Longrightarrow\quad f_k(N)\;\ge\; f_3(N).
\]
In the comments part of \cite{EP856}, Bloom observed that $f_3(N)\gg \log\log N$. A closer inspection of the same construction yields the stronger bound
\[
f_3(N)\;\gg_t\; (\log\log N)^t
\quad\text{for every integer } t\ge1.
\]
Since the detailed proofs in the comment section are rather long, we have
included complete versions of these arguments in the Appendix of this note.

In this note we go significantly further and prove that
\[
f_3(N)\;\gg\; (\log N)^{\,1/e - o(1)}.
\]




\section{Main result}

We recall that $f_3(N)$ denotes the maximum of
\[
\sum_{n\in A}\frac1n
\]
over all subsets $A\subset\{1,\dots,N\}$ which contain no triple
$\{a,b,c\}$ of distinct elements with
\[
\operatorname{lcm}(a,b) = \operatorname{lcm}(a,c) = \operatorname{lcm}(b,c).
\]

\begin{thm}
For every constant $0<c<1/e$, we have
\[
f_3(N) \gg_c (\log N)^c
\]
for all sufficiently large $N$.
\end{thm}

\begin{proof}
Fix $c\in(0,1/e)$ and define
\[
g(\alpha) := \alpha\log\frac1\alpha,\qquad 0<\alpha<1.
\]
It is well known that $g$ attains its maximum $1/e$ at $\alpha=1/e$, and $g$ is continuous
on $(0,1)$.  Hence we may choose $\alpha\in(0,1)$ such that
\[
g(\alpha) > c.
\]
Let $\varepsilon>0$ be such that
\[
g(\alpha) - 2\varepsilon > c.
\]


For each sufficiently large $N$, set
\[
L := \log\log N,\qquad t := \bigl\lfloor \alpha L\bigr\rfloor,
\]
so that $t\to\infty$ and $t = \alpha L + O(1)$, and define
\[
x := N^{1/t}.
\]
Then $x\to\infty$ as $N\to\infty$.  We also define \(P_0 := \lceil t^2\rceil\), and work with primes in the interval $[P_0,x]$.

For $y\ge 2$, let
\[
H(y) := \sum_{p\le y}\frac1p
\]
denote the harmonic sum over primes $\le y$.  By Mertens' theorem,
\[
H(y) = \log\log y + M + O\!\Bigl(\frac1{\log y}\Bigr)
\qquad(y\to\infty),
\]
for some absolute constant $M$.

We first estimate $H(x)$.  Note that
\[
\log\log x
= \log\Bigl(\frac{\log N}{t}\Bigr)
= \log\log N - \log t
= L - \log t.
\]
Since $t=\alpha L+O(1)$, we have $\log t = \log L + O(1)$ and hence
\[
\log\log x = L - \log L + O(1),
\]
so in particular
\[
H(x) = \log\log x + O(1) = L + O(\log L).
\]

Let
\[
H_{>P_0}(x) := \sum_{P_0\le p\le x}\frac1p = H(x) - H(P_0^-),
\]
where $H(P_0^-)=\sum_{p<P_0}1/p$.  By Mertens' theorem applied at $y=P_0$,
\[
H(P_0^-) = \log\log P_0 + O(1)
= \log\log(t^2) + O(1)
= \log\log t + O(1)
= O(\log\log L),
\]
so
\[
H_{>P_0}(x)
= L + O(\log L).
\]
Consequently,
\begin{equation}\label{eq:Hratio}
\frac{H_{>P_0}(x)}{t}
= \frac{L + O(\log L)}{\alpha L + O(1)}
= \frac1\alpha + O\Bigl(\frac{\log L}{L}\Bigr)
= \frac1\alpha + o(1)
\qquad(N\to\infty).
\end{equation}


List the primes in $[P_0,x]$ in increasing order as
\[
q_1 < q_2 < \dots < q_M \le x.
\]
We now partition $\{q_1,\dots,q_M\}$ into $t$ disjoint subsets
$P_1,\dots,P_t$ by the following greedy algorithm.

Initialize $P_i:=\varnothing$ and $S_i:=0$ for $1\le i\le t$.
For $j=1,2,\dots,M$, having already assigned $q_1,\dots,q_{j-1}$ to some of the
$P_i$, choose an index $i_j\in\{1,\dots,t\}$ such that $S_{i_j}$ is minimal,
and put $q_j\in P_{i_j}$, updating $S_{i_j}\gets S_{i_j} + 1/q_j$.

For each $i$ we write
\[
S_i(x) := \sum_{p\in P_i}\frac1p.
\]
We claim that for every $N$,
\begin{equation}\label{eq:balance}
\max_{1\le i\le t} S_i(x) - \min_{1\le i\le t} S_i(x) \;\le\; \frac1{P_0}.
\end{equation}
This is proved by induction on $j$.  Let $S_i^{(j)}$ denote the sums after
the first $j$ primes have been assigned, and let $M_j:=\max_i S_i^{(j)}$,
$m_j:=\min_i S_i^{(j)}$.  Initially $M_0=m_0=0$.  Suppose $M_{j-1}-m_{j-1}\le 1/P_0$.
When we assign $q_j$ of weight $w_j=1/q_j\le 1/P_0$ to a bucket $i_j$ with
$S_{i_j}^{(j-1)}=m_{j-1}$, we obtain $S_{i_j}^{(j)} = m_{j-1}+w_j$ and
all other $S_i^{(j)}$ remain unchanged.  Thus $m_j\ge m_{j-1}$ and
\[
M_j = \max\{M_{j-1},\,m_{j-1}+w_j\}.
\]
If $M_{j-1}\ge m_{j-1}+w_j$, then $M_j=M_{j-1}$ and
\[
M_j - m_j \le M_{j-1}-m_{j-1} \le \frac1{P_0}.
\]
If $M_{j-1}< m_{j-1}+w_j$, then $M_j = m_{j-1}+w_j$ and
\[
M_j - m_j \le (m_{j-1}+w_j) - m_{j-1} = w_j \le \frac1{P_0}.
\]
In either case $M_j-m_j\le 1/P_0$, completing the induction and proving
\eqref{eq:balance}.

Let $m(x):=\min_i S_i(x)$.
Since $H_{>P_0}(x) = \sum_{i=1}^t S_i(x) \ge t m(x)$, we have
\[
m(x) \le \frac{H_{>P_0}(x)}{t}.
\]
Combining this with \eqref{eq:balance} yields, for every $i$,
\begin{equation}\label{eq:Si-lower}
S_i(x) \;\ge\; m(x) \;\ge\; \frac{H_{>P_0}(x)}{t} - \frac1{P_0}.
\end{equation}
Using \eqref{eq:Hratio} and the fact that $P_0\ge t^2$, hence $1/P_0\le 1/t^2\to 0$,
we deduce that
\begin{equation}\label{eq:Si-limit}
S_i(x) \;\ge\; \frac1\alpha + o(1)
\qquad(N\to\infty),
\end{equation}
uniformly in $1\le i\le t$.

By continuity of the logarithm, there exists $\delta>0$ such that
\[
\log\Bigl(\frac1\alpha - \delta\Bigr) \;\ge\; \log\frac1\alpha - \varepsilon.
\]
From \eqref{eq:Si-limit} we have $S_i(x)\ge \frac1\alpha -\delta$ for all
$1\le i\le t$ and all sufficiently large $N$, and thus
\begin{equation}\label{eq:Si-exp}
\log S_i(x) \;\ge\; \log\frac1\alpha - \varepsilon
\qquad(1\le i\le t)
\end{equation}
for all sufficiently large $N$.



Define
\[
A_N := \Bigl\{\, n = p_1\cdots p_t :
   p_i\in P_i\ \text{for }1\le i\le t\,\Bigr\}.
\]
Each $p_i$ is a prime $\le x$, so any $n\in A_N$ satisfies
\[
n \le x^t = N,
\]
and hence $A_N\subset\{1,\dots,N\}$.

We claim that $A_N$ contains no triple $\{a,b,c\}$ of distinct elements with all
pairwise least common multiples equal, so $A_N$ is admissible in the definition
of $f_3(N)$.  Indeed, write
\[
a = \prod_{i=1}^t p_i^{(a)},\quad
b = \prod_{i=1}^t p_i^{(b)},\quad
c = \prod_{i=1}^t p_i^{(c)},
\]
where $p_i^{(\cdot)}\in P_i$ for each $i$.  For a fixed $i$, consider the primes
from $P_i$ appearing in the pairwise least common multiples.  In
$\operatorname{lcm}(a,b)$ the primes from $P_i$ form the set
$\{p_i^{(a)},p_i^{(b)}\}$ (with multiplicities ignored), and similarly for
$\operatorname{lcm}(a,c)$ and $\operatorname{lcm}(b,c)$ we obtain the sets
$\{p_i^{(a)},p_i^{(c)}\}$ and $\{p_i^{(b)},p_i^{(c)}\}$ respectively.

If
\[
\operatorname{lcm}(a,b)
= \operatorname{lcm}(a,c)
= \operatorname{lcm}(b,c),
\]
then for each $i$ these three sets of primes from $P_i$ must be equal:
\[
\{p_i^{(a)},p_i^{(b)}\}
= \{p_i^{(a)},p_i^{(c)}\}
= \{p_i^{(b)},p_i^{(c)}\}.
\]
A simple check shows that three two-element sets of the form
$\{x,y\},\{x,z\},\{y,z\}$ can be equal only if $x=y=z$.  Thus
$p_i^{(a)}=p_i^{(b)}=p_i^{(c)}$ for each $i$, and hence $a=b=c$, contradicting
distinctness.  Therefore $A_N$ is admissible, and
\[
f_3(N) \;\ge\; \sum_{n\in A_N}\frac1n.
\]



We have
\[
\sum_{n\in A_N}\frac1n
= \sum_{p_1\in P_1}\cdots\sum_{p_t\in P_t}
  \frac{1}{p_1\cdots p_t}
= \prod_{i=1}^t \sum_{p\in P_i}\frac1p
= \prod_{i=1}^t S_i(x).
\]
Taking logarithms and using \eqref{eq:Si-exp}, we obtain, for all sufficiently
large $N$,
\[
\log\Bigl(\sum_{n\in A_N}\frac1n\Bigr)
= \sum_{i=1}^t \log S_i(x)
\;\ge\; t\Bigl(\log\frac1\alpha - \varepsilon\Bigr).
\]
Recalling that $t=\alpha L + O(1)$ with $L=\log\log N$, we deduce
\[
\log\Bigl(\sum_{n\in A_N}\frac1n\Bigr)
\;\ge\; (\alpha L + O(1))\Bigl(\log\frac1\alpha - \varepsilon\Bigr)
= \bigl(\alpha\log\tfrac1\alpha - \alpha\varepsilon + o(1)\bigr)L
\]
as $N\to\infty$.  Hence, for all sufficiently large $N$,
\[
\sum_{n\in A_N}\frac1n
\;\ge\;
\exp\bigl((g(\alpha)-2\varepsilon)L\bigr)
= (\log N)^{\,g(\alpha)-2\varepsilon}.
\]

Since $g(\alpha)-2\varepsilon>c$, we have
\[
(\log N)^{\,g(\alpha)-2\varepsilon}
= (\log N)^c\,(\log N)^{\,g(\alpha)-2\varepsilon-c}
\ge (\log N)^c
\]
for all sufficiently large $N$ (because $g(\alpha)-2\varepsilon-c>0$ is fixed).
Thus there exists $N_0(c)$ such that for all $N\ge N_0(c)$,
\[
f_3(N) \;\ge\; \sum_{n\in A_N}\frac1n
\;\gg_c\; (\log N)^c,
\]
where the implied constant may depend on $c$ but not on $N$.  This proves the theorem.
\end{proof}




\begin{thebibliography}{99}
\bibitem{EP856}
T. F. Bloom, Erdős Problem \#856, \url{https://www.erdosproblems.com/856}, accessed 2025-12-10.

\bibitem{Er70}
Erd\H{o}s, Paul, Some extremal problems in combinatorial number theory. Mathematical Essays Dedicated to A. J. Macintyre (1970), 123--133.

\end{thebibliography}










\newpage


\appendix
\section{Comments (15:18 on 09 Dec 2025)}\label{sec:comment1}
For $k=3$ one can obtain a lower bound stronger than $\gg \log\log N$ from Bloom's comment (12:38 on 09 Dec 2025) by looking at products of two primes from different congruence classes.  Fix, for instance,\[
P_1 := \{p \text{ prime} : p \equiv 1 \pmod 8\}, \qquad
P_2 := \{q \text{ prime} : q \equiv 3 \pmod 8\},
\]and for each $N$ set\[
A_N := \{\, n = pq \le N : p \in P_1,\ q \in P_2 \,\}.
\]I claim that $A_N$ contains no triple $\{a,b,c\}$ with all pairwise least common multiples equal.  Indeed, write\[
a = p_1 q_1,\quad b=p_2 q_2,\quad c=p_3 q_3
\]with $p_i\in P_1$ and $q_i\in P_2$.  In the prime factorization of $\operatorname{lcm}(a,b)$, the contribution from primes in $P_1$ is exactly the set $\{p_1,p_2\}$, and similarly for the other two lcm's we obtain the sets $\{p_1,p_3\}$ and $\{p_2,p_3\}$.  These three two-element sets can only be equal if $p_1=p_2=p_3$, and the same argument applied to the primes in $P_2$ forces $q_1=q_2=q_3$, hence $a=b=c$.  Thus $A_N$ is admissible for the definition of $f_3(N)$.
Now
\[
f_3(N)\geq\sum_{n\in A_N}\frac1n
= \sum_{\substack{p\in P_1\\ q\in P_2\\ pq\le N}} \frac1{pq}
\ge \sum_{\substack{p\in P_1\\ p\le \sqrt N}} \frac1p
     \sum_{\substack{q\in P_2\\ q\le N/p}} \frac1q.
\]By Mertens' theorem in arithmetic progressions one has
\[
\sum_{\substack{p\le x\\ p\equiv a\ (\mathrm{mod}\ 8)}} \frac1p
= \frac{1}{\varphi(8)}\log\log x + O(1)
= \frac14\log\log x + O(1)
\]for each residue class $a$ coprime to $8$.  In particular, there exists a constant $c_0>0$ and $x_0\ge 2$ such that, for all $x\ge x_0$ and all $a\in\{1,3,5,7\}$,
\[
\sum_{\substack{r\le x\\ r\equiv a\ (\mathrm{mod}\ 8)}} \frac1r
\;\ge\; c_0 \log\log x.
\]Assume $N$ is large enough that $\sqrt N \ge x_0$.  Then, for any prime $p\in P_1$ with $p\le \sqrt N$, we have $N/p \ge \sqrt N \ge x_0$, so
\[
\sum_{\substack{q\in P_2\\ q\le N/p}} \frac1q 
\;\ge\; c_0 \log\log\!\Bigl(\frac{N}{p}\Bigr).
\]
Since $p\le\sqrt N$, we have $N/p \in [\sqrt N,\,N]$, and hence
\[
\log\log\!\Bigl(\frac{N}{p}\Bigr)
\;\ge\; \log\log\sqrt N
= \log\log N - \log 2.
\]
Thus, enlarging $N$ further if necessary, we may absorb the additive constant $\log 2$ into the implicit constant and obtain
\[
\sum_{\substack{q\in P_2\\ q\le N/p}} \frac1q \;\gg\; \log\log N
\qquad\text{for all }p\in P_1\text{ with }p\le\sqrt N.
\]Similarly, applying the same Mertens-type estimate with $x=\sqrt N$ and $a=1$ gives
\[
\sum_{\substack{p\in P_1\\ p\le \sqrt N}} \frac1p
\;\ge\; c_0 \log\log\sqrt N
= c_0(\log\log N - \log 2)
\;\gg\; \log\log N
\]
for all sufficiently large $N$.  Combining these lower bounds, we deduce
\[
f_3(N)\;\geq\;\sum_{n\in A_N}\frac1n
\;\gg\; (\log\log N)\cdot(\log\log N)
\;=\; (\log\log N)^2.
\]








\section{Comments (15:30 on 09 Dec 2025)}\label{sec:comment2}

I think the above construction extends in a straightforward way to give higher powers of $\log\log N$. Indeed, fix an integer $t\ge1$ and choose $t$ distinct residue classes $a_1,\dots,a_t \pmod q$ with $(a_i,q)=1$. For each $1\le i\le t$ let 
\[
P_i := \{p \text{ prime} : p \equiv a_i \pmod q\},
\]and define
\[
A_N^{(t)} := \bigl\{\, n=p_1\cdots p_t \le N : p_i\in P_i\ (1\le i\le t) \,\bigr\}.
\]By the same argument as in the case $t=2$, one checks that $A_N^{(t)}$ contains no triple $\{a,b,c\}$ with all pairwise least common multiples equal. Furthermore, using Mertens' theorem in arithmetic progressions in each of the $t$ sets $P_i$ and restricting to the box $p_i\le N^{1/t}$, one obtains
\[
\sum_{n\in A_N^{(t)}} \frac1n
\;\ge\;
\prod_{i=1}^t \sum_{\substack{p\in P_i\\ p\le N^{1/t}}} \frac1p
\;\gg_t\; (\log\log N)^t
\]for all sufficiently large $N$, where the implied constant depends only on $t$. Hence for every fixed $t$ there exists a constant $c_t>0$ such that
\[
f_3(N) \;\ge\; c_t (\log\log N)^t
\]for all sufficiently large $N$.








\section{Comments (08:19 on 10 Dec 2025)}\label{sec:comment3}


Thanks to Bloom for the very helpful heuristic comment (17:24 on 09 Dec 2025). Motivated by his suggestion that choosing $t$ to be a small multiple of $\log\log N$ might lead to a power of $\log N$, one can in fact turn this into a completely rigorous construction.


\begin{thm}
There exist constants $c>0$ such that
\[
f_3(N) \gg_c (\log N)^c
\]for all sufficiently large $N$.    
\end{thm} 

\begin{proof}
Fix a real number $B>1$, and choose a small parameter $\gamma\in(0,B/2)$. Set
\[
B' := B+\gamma.
\]Next choose a real number $P_0>0$ so large that for every prime $p\ge P_0$ we have
\[
\frac{1}{p} \le \gamma.
\](This is possible since $1/p\to 0$ as $p\to\infty$.)

We will work with primes $\ge P_0$, and we write
\[
H(x) := \sum_{p\le x}\frac{1}{p}
\]for the usual harmonic sum over primes. By Mertens' theorem we have
\[
H(x) = \log\log x + O(1)
\qquad(x\to\infty).
\]Let us now choose a small constant \(\kappa := \frac{1}{16B'}\). For each large $N$ we set $L := \log\log N$, $t := \big\lfloor \kappa L \big\rfloor$, and define \(x := N^{1/t}\). For $N$ sufficiently large we have $t\ge 1$ and $x\to\infty$.

First, we have
\[
\log\log x
= \log\Bigl(\frac{\log N}{t}\Bigr)
= \log\log N - \log t
= L - \log t.
\]Since $t=\kappa L+O(1)$, we have $\log t = \log L + O(1)$, so
\[
\log\log x
= L - (\log L + O(1))
\ge \frac{L}{2}
\]for all sufficiently large $N$. Hence Mertens' theorem gives
\[
H(x)
= \log\log x + O(1)
\ge \frac{L}{2} - C_0
\]for some constant $C_0$ and all large $N$.

Let
\[
S_{\text{tot}}(x)
:= \sum_{P_0 \le p\le x}\frac{1}{p}
= H(x) - H(P_0^-),
\]where $H(P_0^-)$ is a fixed constant depending only on $P_0$. Thus
\[
S_{\text{tot}}(x)
\ge \frac{L}{2} - C_1
\]for some constant $C_1$ and all large $N$. In particular, there exists $N_1$ such that for all $N\ge N_1$,
\[
S_{\text{tot}}(x) \;\ge\; \frac{L}{4}. \qquad \text{(eq:Stot-large)}
\]On the other hand
\[
tB' \le (\kappa L)B' \le \frac{L}{16}
\]for all large $N$, since $\kappa=1/(16B')$ and $t\le\kappa L$. Thus, by Eq. (eq:Stot-large), for all $N\ge N_1$ we have
\[
S_{\text{tot}}(x) \;\ge\; tB'. \qquad \text{(eq:Stot-vs-tBprime)}
\]

List the primes in $[P_0,x]$ in increasing order as
\[
q_1 < q_2 < \dots < q_M \le x.
\]
We will construct, by a greedy algorithm, disjoint subsets $P_1,\dots,P_k\subset\{q_1,\dots,q_M\}$, where $k\ge t$, such that each $P_i$ satisfies
\[
B \;\le\; \sum_{p\in P_i}\frac{1}{p} \;<\; B'. \qquad \text{(eq:Pi-bounds)}
\]We proceed as follows. Set $P_1=\varnothing$ and $i=1$. Process the primes $q_1,q_2,\dots,q_M$ in order, adding $q_j$ to the current set $P_i$ as long as
\[
\sum_{p\in P_i}\frac{1}{p} < B.
\]As soon as this sum first becomes $\ge B$, we stop filling $P_i$ and, if $i<t$, we start a new set $P_{i+1}$ from the next unused prime in the list. We continue in this way until either we have formed $t$ such sets or we have exhausted all primes in $[P_0,x]$.

By construction, whenever a set $P_i$ is completed, its harmonic sum satisfies $\sum_{p\in P_i}1/p\ge B$, and since each added prime $p$ satisfies $p\ge P_0$ and hence $1/p\le\gamma$, we have
\[
\sum_{p\in P_i}\frac{1}{p} < B + \gamma = B'.
\]
Thus Eq. (eq:Pi-bounds) holds for every completed set $P_i$. 

We now show that for $N\ge N_1$ the algorithm must produce at least $t$ such sets. Suppose instead that the process terminates after forming only $k<t$ completed sets $P_1,\dots,P_k$. There are two possibilities:
Case 1: There is no unfinished set at termination. Then all primes in $[P_0,x]$ have been used to form the sets $P_1,\dots,P_k$, and
\[
S_{\text{tot}}(x)
= \sum_{i=1}^k \sum_{p\in P_i}\frac{1}{p}.
\]
Using Eq. (eq:Pi-bounds), we obtain
\[
S_{\text{tot}}(x)
< \sum_{i=1}^k B'
= kB'
\le (t-1)B' < tB'.
\]Case 2: There is an unfinished $(k+1)$-st set at termination. In this case, after forming $P_1,\dots,P_k$, there remain some primes in $[P_0,x]$ which have been partially used to form a final set $P_{k+1}$ with
\[
\sum_{p\in P_{k+1}}\frac{1}{p} < B,
\]since otherwise $P_{k+1}$ would also be a completed set. Let $P_{k+1}$ denote this (possibly empty) set of remaining primes. Then
\[
S_{\text{tot}}(x)
= \sum_{i=1}^k \sum_{p\in P_i}\frac{1}{p}
  + \sum_{p\in P_{k+1}}\frac{1}{p}.
\]
Using Eq. (eq:Pi-bounds) again, we obtain
\[
S_{\text{tot}}(x)
< kB' + B
\le kB' + B'
= (k+1)B'
\le tB'.
\]In both cases we conclude that $S_{\text{tot}}(x) < tB'$, which contradicts Eq. (eq:Stot-vs-tBprime). Therefore, for every $N\ge N_1$ the greedy algorithm produces at least $t$ disjoint subsets $P_1,\dots,P_t$ of primes in $[P_0,x]$, each satisfying Eq. (eq:Pi-bounds).

Now define
\[
A_N := \Bigl\{\, n = p_1\cdots p_t :
   p_i\in P_i\ \text{for }1\le i\le t\,\Bigr\}.
\]Since every prime in $P_i$ is $\le x$, any $n\in A_N$ satisfies
\[
n \le x^t = N,
\]so $A_N\subset\{1,\dots,N\}$. Clearly, $A_N$ contains no triple $\{a,b,c\}$ of distinct elements with all pairwise least common multiples equal.


Finally, we have
\[
\sum_{n\in A_N}\frac{1}{n}
= \sum_{p_1\in P_1}\cdots\sum_{p_t\in P_t}\frac{1}{p_1\cdots p_t}
= \prod_{i=1}^t \sum_{p\in P_i}\frac{1}{p}.
\]By Eq. (eq:Pi-bounds), each factor satisfies
\[
\sum_{p\in P_i}\frac{1}{p} \;\ge\; B,
\]so
\[
\sum_{n\in A_N}\frac{1}{n}
\;\ge\; B^t.
\]Since $A_N$ is admissible, this implies
\[
f_3(N) \;\ge\; B^t.
\]Recalling that $t=\lfloor\kappa L\rfloor$ with $L=\log\log N$, we obtain
\[
B^t \;\ge\; B^{\kappa L - 1}
= \exp\bigl( (\kappa L - 1)\log B \bigr)
= (\log N)^{\kappa\log B}\, B^{-1}.
\]Thus there exists $c:=\kappa\log B>0$ and a constant $C>0$ (depending only on $B,\gamma,P_0$) such that
\[
f_3(N) \;\ge\; C\,(\log N)^c
\]for all sufficiently large $N$. Q.E.D.
\end{proof}

For example, if we let $B=2$ and $\gamma=\tfrac{1}{2}$ in the proof of Theorem 1, then $B' = B + \gamma = \frac{5}{2}$ and $\kappa = \frac{1}{16B'} = \frac{1}{40}$. Thus 
\[
f_3(N) \gg (\log N)^c, \qquad \text{where } c = \frac{1}{40}\log 2 > 0.017.
\]Of course, with a more refined construction one can improve this value of $c$, but it seems that this method can at best yield exponents in the range $0 < c < 1/e$.














\end{document}
